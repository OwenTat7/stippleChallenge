% Options for packages loaded elsewhere
% Options for packages loaded elsewhere
\PassOptionsToPackage{unicode}{hyperref}
\PassOptionsToPackage{hyphens}{url}
\PassOptionsToPackage{dvipsnames,svgnames,x11names}{xcolor}
%
\documentclass[
  letterpaper,
  DIV=11,
  numbers=noendperiod]{scrartcl}
\usepackage{xcolor}
\usepackage{amsmath,amssymb}
\setcounter{secnumdepth}{-\maxdimen} % remove section numbering
\usepackage{iftex}
\ifPDFTeX
  \usepackage[T1]{fontenc}
  \usepackage[utf8]{inputenc}
  \usepackage{textcomp} % provide euro and other symbols
\else % if luatex or xetex
  \usepackage{unicode-math} % this also loads fontspec
  \defaultfontfeatures{Scale=MatchLowercase}
  \defaultfontfeatures[\rmfamily]{Ligatures=TeX,Scale=1}
\fi
\usepackage{lmodern}
\ifPDFTeX\else
  % xetex/luatex font selection
\fi
% Use upquote if available, for straight quotes in verbatim environments
\IfFileExists{upquote.sty}{\usepackage{upquote}}{}
\IfFileExists{microtype.sty}{% use microtype if available
  \usepackage[]{microtype}
  \UseMicrotypeSet[protrusion]{basicmath} % disable protrusion for tt fonts
}{}
\makeatletter
\@ifundefined{KOMAClassName}{% if non-KOMA class
  \IfFileExists{parskip.sty}{%
    \usepackage{parskip}
  }{% else
    \setlength{\parindent}{0pt}
    \setlength{\parskip}{6pt plus 2pt minus 1pt}}
}{% if KOMA class
  \KOMAoptions{parskip=half}}
\makeatother
% Make \paragraph and \subparagraph free-standing
\makeatletter
\ifx\paragraph\undefined\else
  \let\oldparagraph\paragraph
  \renewcommand{\paragraph}{
    \@ifstar
      \xxxParagraphStar
      \xxxParagraphNoStar
  }
  \newcommand{\xxxParagraphStar}[1]{\oldparagraph*{#1}\mbox{}}
  \newcommand{\xxxParagraphNoStar}[1]{\oldparagraph{#1}\mbox{}}
\fi
\ifx\subparagraph\undefined\else
  \let\oldsubparagraph\subparagraph
  \renewcommand{\subparagraph}{
    \@ifstar
      \xxxSubParagraphStar
      \xxxSubParagraphNoStar
  }
  \newcommand{\xxxSubParagraphStar}[1]{\oldsubparagraph*{#1}\mbox{}}
  \newcommand{\xxxSubParagraphNoStar}[1]{\oldsubparagraph{#1}\mbox{}}
\fi
\makeatother

\usepackage{color}
\usepackage{fancyvrb}
\newcommand{\VerbBar}{|}
\newcommand{\VERB}{\Verb[commandchars=\\\{\}]}
\DefineVerbatimEnvironment{Highlighting}{Verbatim}{commandchars=\\\{\}}
% Add ',fontsize=\small' for more characters per line
\usepackage{framed}
\definecolor{shadecolor}{RGB}{241,243,245}
\newenvironment{Shaded}{\begin{snugshade}}{\end{snugshade}}
\newcommand{\AlertTok}[1]{\textcolor[rgb]{0.68,0.00,0.00}{#1}}
\newcommand{\AnnotationTok}[1]{\textcolor[rgb]{0.37,0.37,0.37}{#1}}
\newcommand{\AttributeTok}[1]{\textcolor[rgb]{0.40,0.45,0.13}{#1}}
\newcommand{\BaseNTok}[1]{\textcolor[rgb]{0.68,0.00,0.00}{#1}}
\newcommand{\BuiltInTok}[1]{\textcolor[rgb]{0.00,0.23,0.31}{#1}}
\newcommand{\CharTok}[1]{\textcolor[rgb]{0.13,0.47,0.30}{#1}}
\newcommand{\CommentTok}[1]{\textcolor[rgb]{0.37,0.37,0.37}{#1}}
\newcommand{\CommentVarTok}[1]{\textcolor[rgb]{0.37,0.37,0.37}{\textit{#1}}}
\newcommand{\ConstantTok}[1]{\textcolor[rgb]{0.56,0.35,0.01}{#1}}
\newcommand{\ControlFlowTok}[1]{\textcolor[rgb]{0.00,0.23,0.31}{\textbf{#1}}}
\newcommand{\DataTypeTok}[1]{\textcolor[rgb]{0.68,0.00,0.00}{#1}}
\newcommand{\DecValTok}[1]{\textcolor[rgb]{0.68,0.00,0.00}{#1}}
\newcommand{\DocumentationTok}[1]{\textcolor[rgb]{0.37,0.37,0.37}{\textit{#1}}}
\newcommand{\ErrorTok}[1]{\textcolor[rgb]{0.68,0.00,0.00}{#1}}
\newcommand{\ExtensionTok}[1]{\textcolor[rgb]{0.00,0.23,0.31}{#1}}
\newcommand{\FloatTok}[1]{\textcolor[rgb]{0.68,0.00,0.00}{#1}}
\newcommand{\FunctionTok}[1]{\textcolor[rgb]{0.28,0.35,0.67}{#1}}
\newcommand{\ImportTok}[1]{\textcolor[rgb]{0.00,0.46,0.62}{#1}}
\newcommand{\InformationTok}[1]{\textcolor[rgb]{0.37,0.37,0.37}{#1}}
\newcommand{\KeywordTok}[1]{\textcolor[rgb]{0.00,0.23,0.31}{\textbf{#1}}}
\newcommand{\NormalTok}[1]{\textcolor[rgb]{0.00,0.23,0.31}{#1}}
\newcommand{\OperatorTok}[1]{\textcolor[rgb]{0.37,0.37,0.37}{#1}}
\newcommand{\OtherTok}[1]{\textcolor[rgb]{0.00,0.23,0.31}{#1}}
\newcommand{\PreprocessorTok}[1]{\textcolor[rgb]{0.68,0.00,0.00}{#1}}
\newcommand{\RegionMarkerTok}[1]{\textcolor[rgb]{0.00,0.23,0.31}{#1}}
\newcommand{\SpecialCharTok}[1]{\textcolor[rgb]{0.37,0.37,0.37}{#1}}
\newcommand{\SpecialStringTok}[1]{\textcolor[rgb]{0.13,0.47,0.30}{#1}}
\newcommand{\StringTok}[1]{\textcolor[rgb]{0.13,0.47,0.30}{#1}}
\newcommand{\VariableTok}[1]{\textcolor[rgb]{0.07,0.07,0.07}{#1}}
\newcommand{\VerbatimStringTok}[1]{\textcolor[rgb]{0.13,0.47,0.30}{#1}}
\newcommand{\WarningTok}[1]{\textcolor[rgb]{0.37,0.37,0.37}{\textit{#1}}}

\usepackage{longtable,booktabs,array}
\usepackage{calc} % for calculating minipage widths
% Correct order of tables after \paragraph or \subparagraph
\usepackage{etoolbox}
\makeatletter
\patchcmd\longtable{\par}{\if@noskipsec\mbox{}\fi\par}{}{}
\makeatother
% Allow footnotes in longtable head/foot
\IfFileExists{footnotehyper.sty}{\usepackage{footnotehyper}}{\usepackage{footnote}}
\makesavenoteenv{longtable}
\usepackage{graphicx}
\makeatletter
\newsavebox\pandoc@box
\newcommand*\pandocbounded[1]{% scales image to fit in text height/width
  \sbox\pandoc@box{#1}%
  \Gscale@div\@tempa{\textheight}{\dimexpr\ht\pandoc@box+\dp\pandoc@box\relax}%
  \Gscale@div\@tempb{\linewidth}{\wd\pandoc@box}%
  \ifdim\@tempb\p@<\@tempa\p@\let\@tempa\@tempb\fi% select the smaller of both
  \ifdim\@tempa\p@<\p@\scalebox{\@tempa}{\usebox\pandoc@box}%
  \else\usebox{\pandoc@box}%
  \fi%
}
% Set default figure placement to htbp
\def\fps@figure{htbp}
\makeatother





\setlength{\emergencystretch}{3em} % prevent overfull lines

\providecommand{\tightlist}{%
  \setlength{\itemsep}{0pt}\setlength{\parskip}{0pt}}



 


\KOMAoption{captions}{tableheading}
\makeatletter
\@ifpackageloaded{tcolorbox}{}{\usepackage[skins,breakable]{tcolorbox}}
\@ifpackageloaded{fontawesome5}{}{\usepackage{fontawesome5}}
\definecolor{quarto-callout-color}{HTML}{909090}
\definecolor{quarto-callout-note-color}{HTML}{0758E5}
\definecolor{quarto-callout-important-color}{HTML}{CC1914}
\definecolor{quarto-callout-warning-color}{HTML}{EB9113}
\definecolor{quarto-callout-tip-color}{HTML}{00A047}
\definecolor{quarto-callout-caution-color}{HTML}{FC5300}
\definecolor{quarto-callout-color-frame}{HTML}{acacac}
\definecolor{quarto-callout-note-color-frame}{HTML}{4582ec}
\definecolor{quarto-callout-important-color-frame}{HTML}{d9534f}
\definecolor{quarto-callout-warning-color-frame}{HTML}{f0ad4e}
\definecolor{quarto-callout-tip-color-frame}{HTML}{02b875}
\definecolor{quarto-callout-caution-color-frame}{HTML}{fd7e14}
\makeatother
\makeatletter
\@ifpackageloaded{caption}{}{\usepackage{caption}}
\AtBeginDocument{%
\ifdefined\contentsname
  \renewcommand*\contentsname{Table of contents}
\else
  \newcommand\contentsname{Table of contents}
\fi
\ifdefined\listfigurename
  \renewcommand*\listfigurename{List of Figures}
\else
  \newcommand\listfigurename{List of Figures}
\fi
\ifdefined\listtablename
  \renewcommand*\listtablename{List of Tables}
\else
  \newcommand\listtablename{List of Tables}
\fi
\ifdefined\figurename
  \renewcommand*\figurename{Figure}
\else
  \newcommand\figurename{Figure}
\fi
\ifdefined\tablename
  \renewcommand*\tablename{Table}
\else
  \newcommand\tablename{Table}
\fi
}
\@ifpackageloaded{float}{}{\usepackage{float}}
\floatstyle{ruled}
\@ifundefined{c@chapter}{\newfloat{codelisting}{h}{lop}}{\newfloat{codelisting}{h}{lop}[chapter]}
\floatname{codelisting}{Listing}
\newcommand*\listoflistings{\listof{codelisting}{List of Listings}}
\makeatother
\makeatletter
\makeatother
\makeatletter
\@ifpackageloaded{caption}{}{\usepackage{caption}}
\@ifpackageloaded{subcaption}{}{\usepackage{subcaption}}
\makeatother
\usepackage{bookmark}
\IfFileExists{xurl.sty}{\usepackage{xurl}}{} % add URL line breaks if available
\urlstyle{same}
\hypersetup{
  pdftitle={Selection Bias \& Missing Data Challenge - Part 1},
  colorlinks=true,
  linkcolor={blue},
  filecolor={Maroon},
  citecolor={Blue},
  urlcolor={Blue},
  pdfcreator={LaTeX via pandoc}}


\title{Selection Bias \& Missing Data Challenge - Part 1}
\usepackage{etoolbox}
\makeatletter
\providecommand{\subtitle}[1]{% add subtitle to \maketitle
  \apptocmd{\@title}{\par {\large #1 \par}}{}{}
}
\makeatother
\subtitle{Blue Noise Stippling: Creating Art from Data}
\author{}
\date{}
\begin{document}
\maketitle


\section{🎨 Selection Bias \& Missing Data Challenge - Part
1}\label{selection-bias-missing-data-challenge---part-1}

\begin{tcolorbox}[enhanced jigsaw, leftrule=.75mm, colback=white, opacityback=0, arc=.35mm, left=2mm, rightrule=.15mm, coltitle=black, titlerule=0mm, breakable, bottomrule=.15mm, colbacktitle=quarto-callout-important-color!10!white, bottomtitle=1mm, toptitle=1mm, colframe=quarto-callout-important-color-frame, title=\textcolor{quarto-callout-important-color}{\faExclamation}\hspace{0.5em}{📊 Challenge Requirements}, toprule=.15mm, opacitybacktitle=0.6]

\textbf{Your Task:} Reproduce the blue noise stippling process
demonstrated below to create: 1. A stippled version of your chosen image
2. A progressive stippling GIF animation 3. Post both to a GitHub Pages
site with appropriate captions and a brief explanation

\textbf{Part 2 Preview:} On November 18th, we'll tackle Part 2 of this
challenge, where you'll create a statistical meme about selection bias
and missing data using your stippled images.

\end{tcolorbox}

\subsection{The Problem: Can Algorithms Create
Art?}\label{the-problem-can-algorithms-create-art}

\textbf{Core Question:} How can we convert a photograph into an
aesthetically pleasing pattern of dots that preserves the visual
information of the original image?

\textbf{The Challenge:} Blue noise stippling is a technique that
converts images into patterns of dots (stipples) using algorithms that
balance visual accuracy with spatial distribution. This challenge asks
you to implement a modified ``void and cluster'' algorithm that combines
importance sampling with blue noise distribution properties to create
stippling patterns that are both visually accurate and spatially
well-distributed.

\textbf{Our Approach:} We'll use a modified void-and-cluster algorithm
that: 1. Creates an importance map identifying visually important
regions 2. Uses a toroidal (periodic) Gaussian kernel for repulsion to
ensure blue noise properties 3. Iteratively selects points with minimum
energy 4. Balances image content importance with blue noise spatial
distribution

\begin{tcolorbox}[enhanced jigsaw, leftrule=.75mm, colback=white, opacityback=0, arc=.35mm, left=2mm, rightrule=.15mm, coltitle=black, titlerule=0mm, breakable, bottomrule=.15mm, colbacktitle=quarto-callout-warning-color!10!white, bottomtitle=1mm, toptitle=1mm, colframe=quarto-callout-warning-color-frame, title=\textcolor{quarto-callout-warning-color}{\faExclamationTriangle}\hspace{0.5em}{⚠️ AI Partnership Required}, toprule=.15mm, opacitybacktitle=0.6]

This challenge pushes boundaries intentionally. You'll tackle problems
that normally require weeks of study, but with Cursor AI as your partner
(and your brain keeping it honest), you can accomplish more than you
thought possible.

\textbf{The new reality:} The four stages of competence are Ignorance →
Awareness → Learning → Mastery. AI lets us produce Mastery-level work
while operating primarily in the Awareness stage. I focus on awareness
training, you leverage AI for execution, and together we create outputs
that used to require years of dedicated study.

\end{tcolorbox}

\subsection{Introduction to Blue Noise
Stippling}\label{introduction-to-blue-noise-stippling}

Blue noise stippling is a technique for converting images into a pattern
of dots (stipples) that preserves the visual information of the original
image while creating an aesthetically pleasing, evenly distributed
pattern. This method follows the approach described by
\href{https://bartwronski.com/2022/08/31/progressive-image-stippling-and-greedy-blue-noise-importance-sampling/}{Bart
Wronski}.

The method uses a modified ``void and cluster'' algorithm that combines
importance sampling with blue noise distribution properties to create
stippling patterns that are both visually accurate and spatially
well-distributed. This version uses \textbf{smooth extreme
downweighting} that selectively downweights very dark and very light
tones while preserving mid-tones, creating a more balanced distribution
of stipples across the image.

\subsection{Loading the Original
Image}\label{loading-the-original-image}

First, let's load an image that we'll convert to a blue noise stippling
pattern. You can use any image you'd like, but we'll demonstrate with
the provided example.

\subsubsection{Python}

\begin{Shaded}
\begin{Highlighting}[]
\ImportTok{import}\NormalTok{ numpy }\ImportTok{as}\NormalTok{ np}
\ImportTok{from}\NormalTok{ PIL }\ImportTok{import}\NormalTok{ Image}
\ImportTok{import}\NormalTok{ matplotlib.pyplot }\ImportTok{as}\NormalTok{ plt}

\CommentTok{\# Load the image}
\NormalTok{img\_path }\OperatorTok{=} \StringTok{\textquotesingle{}IMG\_2198 copy.JPG\textquotesingle{}}
\NormalTok{original\_img }\OperatorTok{=}\NormalTok{ Image.}\BuiltInTok{open}\NormalTok{(img\_path)}

\CommentTok{\# Convert to grayscale if needed}
\ControlFlowTok{if}\NormalTok{ original\_img.mode }\OperatorTok{!=} \StringTok{\textquotesingle{}L\textquotesingle{}}\NormalTok{:}
\NormalTok{    original\_img }\OperatorTok{=}\NormalTok{ original\_img.convert(}\StringTok{\textquotesingle{}L\textquotesingle{}}\NormalTok{)}

\CommentTok{\# Convert to numpy array and normalize to [0, 1]}
\NormalTok{img\_array }\OperatorTok{=}\NormalTok{ np.array(original\_img, dtype}\OperatorTok{=}\NormalTok{np.float32) }\OperatorTok{/} \FloatTok{255.0}

\CommentTok{\# Display the original image}
\NormalTok{fig, ax }\OperatorTok{=}\NormalTok{ plt.subplots(figsize}\OperatorTok{=}\NormalTok{(}\FloatTok{6.5}\NormalTok{, }\DecValTok{5}\NormalTok{))}
\NormalTok{ax.imshow(img\_array, cmap}\OperatorTok{=}\StringTok{\textquotesingle{}gray\textquotesingle{}}\NormalTok{, vmin}\OperatorTok{=}\DecValTok{0}\NormalTok{, vmax}\OperatorTok{=}\DecValTok{1}\NormalTok{)}
\NormalTok{ax.axis(}\StringTok{\textquotesingle{}off\textquotesingle{}}\NormalTok{)}
\NormalTok{ax.set\_title(}\StringTok{\textquotesingle{}Original Image\textquotesingle{}}\NormalTok{, fontsize}\OperatorTok{=}\DecValTok{14}\NormalTok{, fontweight}\OperatorTok{=}\StringTok{\textquotesingle{}bold\textquotesingle{}}\NormalTok{, pad}\OperatorTok{=}\DecValTok{10}\NormalTok{)}
\NormalTok{plt.tight\_layout()}
\NormalTok{plt.show()}

\BuiltInTok{print}\NormalTok{(}\SpecialStringTok{f"Image shape: }\SpecialCharTok{\{}\NormalTok{img\_array}\SpecialCharTok{.}\NormalTok{shape}\SpecialCharTok{\}}\SpecialStringTok{"}\NormalTok{)}
\BuiltInTok{print}\NormalTok{(}\SpecialStringTok{f"Image size: }\SpecialCharTok{\{}\NormalTok{img\_array}\SpecialCharTok{.}\NormalTok{shape[}\DecValTok{0}\NormalTok{]}\SpecialCharTok{\}}\SpecialStringTok{x}\SpecialCharTok{\{}\NormalTok{img\_array}\SpecialCharTok{.}\NormalTok{shape[}\DecValTok{1}\NormalTok{]}\SpecialCharTok{\}}\SpecialStringTok{ pixels"}\NormalTok{)}
\end{Highlighting}
\end{Shaded}

\begin{figure}[H]

{\centering \pandocbounded{\includegraphics[keepaspectratio]{index_files/figure-pdf/load-image-output-1.pdf}}

}

\caption{Original image before stippling}

\end{figure}%

\begin{verbatim}
Image shape: (6328, 4219)
Image size: 6328x4219 pixels
\end{verbatim}

\subsection{Importance Mapping}\label{importance-mapping}

Before applying the stippling algorithm, we create an \textbf{importance
map} that identifies which regions of the image should receive more
stipples. The importance map is computed by:

\begin{itemize}
\tightlist
\item
  \textbf{Brightness inversion}: The image brightness is inverted so
  that dark areas receive higher importance and thus more dots, while
  light areas receive fewer dots
\item
  \textbf{Extreme tone downweighting}: Smooth Gaussian functions
  downweight tones below 0.2 (very dark) and above 0.8 (very light),
  creating a gradual transition that preserves mid-tones
\item
  \textbf{Mid-tone boost}: A smooth Gaussian function centered on
  mid-tones provides a gradual increase in importance for mid-tone
  regions, ensuring they receive appropriate stippling density
\item
  \textbf{Selective and effective}: This approach ensures that stipples
  are distributed appropriately (more dots in dark areas and mid-tones,
  fewer in extreme dark/light areas) while maintaining good spatial
  distribution
\end{itemize}

\subsubsection{Python}

\phantomsection\label{importance-map}
\begin{Shaded}
\begin{Highlighting}[]
\KeywordTok{def}\NormalTok{ compute\_importance(}
\NormalTok{    gray\_img: np.ndarray,}
\NormalTok{    extreme\_downweight: }\BuiltInTok{float} \OperatorTok{=} \FloatTok{0.5}\NormalTok{,}
\NormalTok{    extreme\_threshold\_low: }\BuiltInTok{float} \OperatorTok{=} \FloatTok{0.4}\NormalTok{,}
\NormalTok{    extreme\_threshold\_high: }\BuiltInTok{float} \OperatorTok{=} \FloatTok{0.8}\NormalTok{,}
\NormalTok{    extreme\_sigma: }\BuiltInTok{float} \OperatorTok{=} \FloatTok{0.1}\NormalTok{,}
\NormalTok{    mid\_tone\_boost: }\BuiltInTok{float} \OperatorTok{=} \FloatTok{0.4}\NormalTok{,}
\NormalTok{    mid\_tone\_sigma: }\BuiltInTok{float} \OperatorTok{=} \FloatTok{0.2}\NormalTok{,}
\NormalTok{):}
    \CommentTok{"""}
\CommentTok{    Importance map computation that downweights extreme tones (very dark and very light)}
\CommentTok{    using smooth functions, while boosting mid{-}tones.}
\CommentTok{    }
\CommentTok{    Parameters}
\CommentTok{    {-}{-}{-}{-}{-}{-}{-}{-}{-}{-}}
\CommentTok{    gray\_img : np.ndarray}
\CommentTok{        Grayscale image in [0, 1]}
\CommentTok{    extreme\_downweight : float}
\CommentTok{        Strength of downweighting for extreme tones (0.0 = no downweighting, 1.0 = maximum downweighting)}
\CommentTok{    extreme\_threshold\_low : float}
\CommentTok{        Threshold below which tones are considered "very dark" and get downweighted}
\CommentTok{    extreme\_threshold\_high : float}
\CommentTok{        Threshold above which tones are considered "very light" and get downweighted}
\CommentTok{    extreme\_sigma : float}
\CommentTok{        Width of the smooth transition for extreme downweighting (smaller = sharper transition)}
\CommentTok{    mid\_tone\_boost : float}
\CommentTok{        Strength of mid{-}tone emphasis (0.0 = no boost, 1.0 = strong boost)}
\CommentTok{    mid\_tone\_sigma : float}
\CommentTok{        Width of the mid{-}tone Gaussian bump (smaller = narrower, larger = wider)}
\CommentTok{    }
\CommentTok{    Returns}
\CommentTok{    {-}{-}{-}{-}{-}{-}{-}}
\CommentTok{    importance : np.ndarray}
\CommentTok{        Importance map in [0, 1]; higher = more stipples (dark areas and mid{-}tones get higher importance)}
\CommentTok{    """}
\NormalTok{    I }\OperatorTok{=}\NormalTok{ np.clip(gray\_img, }\FloatTok{0.0}\NormalTok{, }\FloatTok{1.0}\NormalTok{)}
    
    \CommentTok{\# Invert brightness: dark areas should get more dots (higher importance)}
\NormalTok{    I\_inverted }\OperatorTok{=} \FloatTok{1.0} \OperatorTok{{-}}\NormalTok{ I}
    
    \CommentTok{\# Create smooth downweighting mask for extreme tones}
    \CommentTok{\# Downweight very dark regions (I \textless{} extreme\_threshold\_low)}
\NormalTok{    dark\_mask }\OperatorTok{=}\NormalTok{ np.exp(}\OperatorTok{{-}}\NormalTok{((I }\OperatorTok{{-}} \FloatTok{0.0}\NormalTok{) }\OperatorTok{**} \DecValTok{2}\NormalTok{) }\OperatorTok{/}\NormalTok{ (}\FloatTok{2.0} \OperatorTok{*}\NormalTok{ (extreme\_sigma }\OperatorTok{**} \DecValTok{2}\NormalTok{)))}
\NormalTok{    dark\_mask }\OperatorTok{=}\NormalTok{ np.where(I }\OperatorTok{\textless{}}\NormalTok{ extreme\_threshold\_low, dark\_mask, }\FloatTok{0.0}\NormalTok{)}
    \ControlFlowTok{if}\NormalTok{ dark\_mask.}\BuiltInTok{max}\NormalTok{() }\OperatorTok{\textgreater{}} \DecValTok{0}\NormalTok{:}
\NormalTok{        dark\_mask }\OperatorTok{=}\NormalTok{ dark\_mask }\OperatorTok{/}\NormalTok{ dark\_mask.}\BuiltInTok{max}\NormalTok{()}
    
    \CommentTok{\# Downweight very light regions (I \textgreater{} extreme\_threshold\_high)}
\NormalTok{    light\_mask }\OperatorTok{=}\NormalTok{ np.exp(}\OperatorTok{{-}}\NormalTok{((I }\OperatorTok{{-}} \FloatTok{1.0}\NormalTok{) }\OperatorTok{**} \DecValTok{2}\NormalTok{) }\OperatorTok{/}\NormalTok{ (}\FloatTok{2.0} \OperatorTok{*}\NormalTok{ (extreme\_sigma }\OperatorTok{**} \DecValTok{2}\NormalTok{)))}
\NormalTok{    light\_mask }\OperatorTok{=}\NormalTok{ np.where(I }\OperatorTok{\textgreater{}}\NormalTok{ extreme\_threshold\_high, light\_mask, }\FloatTok{0.0}\NormalTok{)}
    \ControlFlowTok{if}\NormalTok{ light\_mask.}\BuiltInTok{max}\NormalTok{() }\OperatorTok{\textgreater{}} \DecValTok{0}\NormalTok{:}
\NormalTok{        light\_mask }\OperatorTok{=}\NormalTok{ light\_mask }\OperatorTok{/}\NormalTok{ light\_mask.}\BuiltInTok{max}\NormalTok{()}
    
    \CommentTok{\# Combine both masks}
\NormalTok{    extreme\_mask }\OperatorTok{=}\NormalTok{ np.maximum(dark\_mask, light\_mask)}
    
    \CommentTok{\# Apply smooth downweighting}
\NormalTok{    importance }\OperatorTok{=}\NormalTok{ I\_inverted }\OperatorTok{*}\NormalTok{ (}\FloatTok{1.0} \OperatorTok{{-}}\NormalTok{ extreme\_downweight }\OperatorTok{*}\NormalTok{ extreme\_mask)}
    
    \CommentTok{\# Add smooth gradual mid{-}tone boost (Gaussian centered on 0.65)}
\NormalTok{    mid\_tone\_center }\OperatorTok{=} \FloatTok{0.65}
\NormalTok{    mid\_tone\_gaussian }\OperatorTok{=}\NormalTok{ np.exp(}\OperatorTok{{-}}\NormalTok{((I }\OperatorTok{{-}}\NormalTok{ mid\_tone\_center) }\OperatorTok{**} \DecValTok{2}\NormalTok{) }\OperatorTok{/}\NormalTok{ (}\FloatTok{2.0} \OperatorTok{*}\NormalTok{ (mid\_tone\_sigma }\OperatorTok{**} \DecValTok{2}\NormalTok{)))}
    \ControlFlowTok{if}\NormalTok{ mid\_tone\_gaussian.}\BuiltInTok{max}\NormalTok{() }\OperatorTok{\textgreater{}} \DecValTok{0}\NormalTok{:}
\NormalTok{        mid\_tone\_gaussian }\OperatorTok{=}\NormalTok{ mid\_tone\_gaussian }\OperatorTok{/}\NormalTok{ mid\_tone\_gaussian.}\BuiltInTok{max}\NormalTok{()}
    
    \CommentTok{\# Boost importance in mid{-}tone regions}
\NormalTok{    importance }\OperatorTok{=}\NormalTok{ importance }\OperatorTok{*}\NormalTok{ (}\FloatTok{1.0} \OperatorTok{+}\NormalTok{ mid\_tone\_boost }\OperatorTok{*}\NormalTok{ mid\_tone\_gaussian)}
    
    \CommentTok{\# Normalize to [0,1]}
\NormalTok{    m, M }\OperatorTok{=}\NormalTok{ importance.}\BuiltInTok{min}\NormalTok{(), importance.}\BuiltInTok{max}\NormalTok{()}
    \ControlFlowTok{if}\NormalTok{ M }\OperatorTok{\textgreater{}}\NormalTok{ m: }
\NormalTok{        importance }\OperatorTok{=}\NormalTok{ (importance }\OperatorTok{{-}}\NormalTok{ m) }\OperatorTok{/}\NormalTok{ (M }\OperatorTok{{-}}\NormalTok{ m)}
    \ControlFlowTok{return}\NormalTok{ importance}
\end{Highlighting}
\end{Shaded}

\subsection{Blue Noise Stippling
Algorithm}\label{blue-noise-stippling-algorithm}

The stippling algorithm uses a modified void-and-cluster approach that:

\begin{enumerate}
\def\labelenumi{\arabic{enumi}.}
\tightlist
\item
  Creates an importance map that identifies visually important regions
\item
  Initializes an energy field based on the importance map (higher
  importance → lower energy)
\item
  Uses a toroidal (periodic) Gaussian kernel for repulsion to ensure
  blue noise properties
\item
  Iteratively selects points with minimum energy
\item
  Adds Gaussian ``splats'' around selected points to prevent clustering
\item
  Balances image content importance with blue noise spatial distribution
\end{enumerate}

\subsubsection{Python}

\phantomsection\label{stippling-functions}
\begin{Shaded}
\begin{Highlighting}[]
\ImportTok{import}\NormalTok{ numpy }\ImportTok{as}\NormalTok{ np}

\KeywordTok{def}\NormalTok{ toroidal\_gaussian\_kernel(h: }\BuiltInTok{int}\NormalTok{, w: }\BuiltInTok{int}\NormalTok{, sigma: }\BuiltInTok{float}\NormalTok{):}
    \CommentTok{"""}
\CommentTok{    Create a periodic (toroidal) 2D Gaussian kernel centered at (0,0).}
\CommentTok{    The toroidal property means the kernel wraps around at the edges,}
\CommentTok{    ensuring consistent repulsion behavior regardless of point location.}
\CommentTok{    """}
\NormalTok{    y }\OperatorTok{=}\NormalTok{ np.arange(h)}
\NormalTok{    x }\OperatorTok{=}\NormalTok{ np.arange(w)}
    \CommentTok{\# Compute toroidal distances (minimum distance considering wrapping)}
\NormalTok{    dy }\OperatorTok{=}\NormalTok{ np.minimum(y, h }\OperatorTok{{-}}\NormalTok{ y)[:, }\VariableTok{None}\NormalTok{]}
\NormalTok{    dx }\OperatorTok{=}\NormalTok{ np.minimum(x, w }\OperatorTok{{-}}\NormalTok{ x)[}\VariableTok{None}\NormalTok{, :]}
    \CommentTok{\# Compute Gaussian}
\NormalTok{    kern }\OperatorTok{=}\NormalTok{ np.exp(}\OperatorTok{{-}}\NormalTok{(dx}\OperatorTok{**}\DecValTok{2} \OperatorTok{+}\NormalTok{ dy}\OperatorTok{**}\DecValTok{2}\NormalTok{) }\OperatorTok{/}\NormalTok{ (}\FloatTok{2.0} \OperatorTok{*}\NormalTok{ sigma}\OperatorTok{**}\DecValTok{2}\NormalTok{))}
\NormalTok{    s }\OperatorTok{=}\NormalTok{ kern.}\BuiltInTok{sum}\NormalTok{()}
    \ControlFlowTok{if}\NormalTok{ s }\OperatorTok{\textgreater{}} \DecValTok{0}\NormalTok{:}
\NormalTok{        kern }\OperatorTok{/=}\NormalTok{ s  }\CommentTok{\# Normalize}
    \ControlFlowTok{return}\NormalTok{ kern}


\KeywordTok{def}\NormalTok{ void\_and\_cluster(}
\NormalTok{    input\_img: np.ndarray,}
\NormalTok{    percentage: }\BuiltInTok{float} \OperatorTok{=} \FloatTok{0.08}\NormalTok{,}
\NormalTok{    sigma: }\BuiltInTok{float} \OperatorTok{=} \FloatTok{0.9}\NormalTok{,}
\NormalTok{    content\_bias: }\BuiltInTok{float} \OperatorTok{=} \FloatTok{0.9}\NormalTok{,}
\NormalTok{    importance\_img: np.ndarray }\OperatorTok{|} \VariableTok{None} \OperatorTok{=} \VariableTok{None}\NormalTok{,}
\NormalTok{    noise\_scale\_factor: }\BuiltInTok{float} \OperatorTok{=} \FloatTok{0.1}\NormalTok{,}
\NormalTok{):}
    \CommentTok{"""}
\CommentTok{    Generate blue noise stippling pattern from input image using a modified}
\CommentTok{    void{-}and{-}cluster algorithm with content{-}weighted importance.}
\CommentTok{    """}
\NormalTok{    I }\OperatorTok{=}\NormalTok{ np.clip(input\_img, }\FloatTok{0.0}\NormalTok{, }\FloatTok{1.0}\NormalTok{)}
\NormalTok{    h, w }\OperatorTok{=}\NormalTok{ I.shape}

    \CommentTok{\# Compute or use provided importance map}
    \ControlFlowTok{if}\NormalTok{ importance\_img }\KeywordTok{is} \VariableTok{None}\NormalTok{:}
\NormalTok{        importance }\OperatorTok{=}\NormalTok{ compute\_importance(I)}
    \ControlFlowTok{else}\NormalTok{:}
\NormalTok{        importance }\OperatorTok{=}\NormalTok{ np.clip(importance\_img, }\FloatTok{0.0}\NormalTok{, }\FloatTok{1.0}\NormalTok{)}

    \CommentTok{\# Create toroidal Gaussian kernel for repulsion}
\NormalTok{    kernel }\OperatorTok{=}\NormalTok{ toroidal\_gaussian\_kernel(h, w, sigma)}

    \CommentTok{\# Initialize energy field: lower energy → more likely to be picked}
\NormalTok{    energy\_current }\OperatorTok{=} \OperatorTok{{-}}\NormalTok{importance }\OperatorTok{*}\NormalTok{ content\_bias}

    \CommentTok{\# Stipple buffer: start with white background; selected points become black dots}
\NormalTok{    final\_stipple }\OperatorTok{=}\NormalTok{ np.ones\_like(I)}
\NormalTok{    samples }\OperatorTok{=}\NormalTok{ []}

    \CommentTok{\# Helper function to roll kernel to an arbitrary position}
    \KeywordTok{def}\NormalTok{ energy\_splat(y, x):}
        \CommentTok{"""Get energy contribution by rolling the kernel to position (y, x)."""}
        \ControlFlowTok{return}\NormalTok{ np.roll(np.roll(kernel, shift}\OperatorTok{=}\NormalTok{y, axis}\OperatorTok{=}\DecValTok{0}\NormalTok{), shift}\OperatorTok{=}\NormalTok{x, axis}\OperatorTok{=}\DecValTok{1}\NormalTok{)}

    \CommentTok{\# Number of points to select}
\NormalTok{    num\_points }\OperatorTok{=} \BuiltInTok{int}\NormalTok{(I.size }\OperatorTok{*}\NormalTok{ percentage)}

    \CommentTok{\# Choose first point near center with minimal energy}
\NormalTok{    cy, cx }\OperatorTok{=}\NormalTok{ h }\OperatorTok{//} \DecValTok{2}\NormalTok{, w }\OperatorTok{//} \DecValTok{2}
\NormalTok{    r }\OperatorTok{=} \BuiltInTok{min}\NormalTok{(}\DecValTok{20}\NormalTok{, h }\OperatorTok{//} \DecValTok{10}\NormalTok{, w }\OperatorTok{//} \DecValTok{10}\NormalTok{)}
\NormalTok{    ys }\OperatorTok{=} \BuiltInTok{slice}\NormalTok{(}\BuiltInTok{max}\NormalTok{(}\DecValTok{0}\NormalTok{, cy }\OperatorTok{{-}}\NormalTok{ r), }\BuiltInTok{min}\NormalTok{(h, cy }\OperatorTok{+}\NormalTok{ r))}
\NormalTok{    xs }\OperatorTok{=} \BuiltInTok{slice}\NormalTok{(}\BuiltInTok{max}\NormalTok{(}\DecValTok{0}\NormalTok{, cx }\OperatorTok{{-}}\NormalTok{ r), }\BuiltInTok{min}\NormalTok{(w, cx }\OperatorTok{+}\NormalTok{ r))}
\NormalTok{    region }\OperatorTok{=}\NormalTok{ energy\_current[ys, xs]}
\NormalTok{    flat }\OperatorTok{=}\NormalTok{ np.argmin(region)}
\NormalTok{    y0 }\OperatorTok{=}\NormalTok{ flat }\OperatorTok{//}\NormalTok{ (region.shape[}\DecValTok{1}\NormalTok{]) }\OperatorTok{+}\NormalTok{ (cy }\OperatorTok{{-}}\NormalTok{ r)}
\NormalTok{    x0 }\OperatorTok{=}\NormalTok{ flat }\OperatorTok{\%}\NormalTok{ (region.shape[}\DecValTok{1}\NormalTok{]) }\OperatorTok{+}\NormalTok{ (cx }\OperatorTok{{-}}\NormalTok{ r)}

    \CommentTok{\# Place first point}
\NormalTok{    energy\_current }\OperatorTok{=}\NormalTok{ energy\_current }\OperatorTok{+}\NormalTok{ energy\_splat(y0, x0)}
\NormalTok{    energy\_current[y0, x0] }\OperatorTok{=}\NormalTok{ np.inf  }\CommentTok{\# Prevent reselection}
\NormalTok{    samples.append((y0, x0, I[y0, x0]))}
\NormalTok{    final\_stipple[y0, x0] }\OperatorTok{=} \FloatTok{0.0}  \CommentTok{\# Black dot}

    \CommentTok{\# Iteratively place remaining points}
    \ControlFlowTok{for}\NormalTok{ i }\KeywordTok{in} \BuiltInTok{range}\NormalTok{(}\DecValTok{1}\NormalTok{, num\_points):}
        \CommentTok{\# Add exploration noise that decreases over time}
\NormalTok{        exploration }\OperatorTok{=} \FloatTok{1.0} \OperatorTok{{-}}\NormalTok{ (i }\OperatorTok{/}\NormalTok{ num\_points) }\OperatorTok{*} \FloatTok{0.5}  \CommentTok{\# Decrease from 1.0 to 0.5}
\NormalTok{        noise }\OperatorTok{=}\NormalTok{ np.random.normal(}\FloatTok{0.0}\NormalTok{, noise\_scale\_factor }\OperatorTok{*}\NormalTok{ content\_bias }\OperatorTok{*}\NormalTok{ exploration, size}\OperatorTok{=}\NormalTok{energy\_current.shape)}
\NormalTok{        energy\_with\_noise }\OperatorTok{=}\NormalTok{ energy\_current }\OperatorTok{+}\NormalTok{ noise}

        \CommentTok{\# Find position with minimum energy (with noise for exploration)}
\NormalTok{        pos\_flat }\OperatorTok{=}\NormalTok{ np.argmin(energy\_with\_noise)}
\NormalTok{        y }\OperatorTok{=}\NormalTok{ pos\_flat }\OperatorTok{//}\NormalTok{ w}
\NormalTok{        x }\OperatorTok{=}\NormalTok{ pos\_flat }\OperatorTok{\%}\NormalTok{ w}

        \CommentTok{\# Add Gaussian splat to prevent nearby points from being selected}
\NormalTok{        energy\_current }\OperatorTok{=}\NormalTok{ energy\_current }\OperatorTok{+}\NormalTok{ energy\_splat(y, x)}
\NormalTok{        energy\_current[y, x] }\OperatorTok{=}\NormalTok{ np.inf  }\CommentTok{\# Prevent reselection}

        \CommentTok{\# Record the sample}
\NormalTok{        samples.append((y, x, I[y, x]))}
\NormalTok{        final\_stipple[y, x] }\OperatorTok{=} \FloatTok{0.0}  \CommentTok{\# Black dot}

    \ControlFlowTok{return}\NormalTok{ final\_stipple, np.array(samples)}
\end{Highlighting}
\end{Shaded}

\subsection{Preparing the Working
Image}\label{preparing-the-working-image}

Before generating the stippling pattern, we prepare the image by
resizing if necessary and computing the importance map.

\subsubsection{Python}

\phantomsection\label{prep-image}
\begin{Shaded}
\begin{Highlighting}[]
\CommentTok{\# Resize image if it\textquotesingle{}s too large for faster processing}
\NormalTok{max\_size }\OperatorTok{=} \DecValTok{512}
\ControlFlowTok{if}\NormalTok{ img\_array.shape[}\DecValTok{0}\NormalTok{] }\OperatorTok{\textgreater{}}\NormalTok{ max\_size }\KeywordTok{or}\NormalTok{ img\_array.shape[}\DecValTok{1}\NormalTok{] }\OperatorTok{\textgreater{}}\NormalTok{ max\_size:}
\NormalTok{    scale }\OperatorTok{=}\NormalTok{ max\_size }\OperatorTok{/} \BuiltInTok{max}\NormalTok{(img\_array.shape[}\DecValTok{0}\NormalTok{], img\_array.shape[}\DecValTok{1}\NormalTok{])}
\NormalTok{    new\_size }\OperatorTok{=}\NormalTok{ (}\BuiltInTok{int}\NormalTok{(img\_array.shape[}\DecValTok{1}\NormalTok{] }\OperatorTok{*}\NormalTok{ scale), }\BuiltInTok{int}\NormalTok{(img\_array.shape[}\DecValTok{0}\NormalTok{] }\OperatorTok{*}\NormalTok{ scale))}
\NormalTok{    img\_resized\_pil }\OperatorTok{=}\NormalTok{ original\_img.resize(new\_size, Image.Resampling.LANCZOS)}
    \ControlFlowTok{if}\NormalTok{ img\_resized\_pil.mode }\OperatorTok{!=} \StringTok{\textquotesingle{}L\textquotesingle{}}\NormalTok{:}
\NormalTok{        img\_resized\_pil }\OperatorTok{=}\NormalTok{ img\_resized\_pil.convert(}\StringTok{\textquotesingle{}L\textquotesingle{}}\NormalTok{)}
\NormalTok{    img\_resized }\OperatorTok{=}\NormalTok{ np.array(img\_resized\_pil, dtype}\OperatorTok{=}\NormalTok{np.float32) }\OperatorTok{/} \FloatTok{255.0}
    \BuiltInTok{print}\NormalTok{(}\SpecialStringTok{f"Resized image from }\SpecialCharTok{\{}\NormalTok{img\_array}\SpecialCharTok{.}\NormalTok{shape}\SpecialCharTok{\}}\SpecialStringTok{ to }\SpecialCharTok{\{}\NormalTok{img\_resized}\SpecialCharTok{.}\NormalTok{shape}\SpecialCharTok{\}}\SpecialStringTok{ for processing"}\NormalTok{)}
\ControlFlowTok{else}\NormalTok{:}
\NormalTok{    img\_resized }\OperatorTok{=}\NormalTok{ img\_array.copy()}

\CommentTok{\# Ensure img\_resized is 2D grayscale}
\ControlFlowTok{if} \BuiltInTok{len}\NormalTok{(img\_resized.shape) }\OperatorTok{\textgreater{}} \DecValTok{2}\NormalTok{:}
\NormalTok{    img\_resized }\OperatorTok{=}\NormalTok{ img\_resized[:, :, }\DecValTok{0}\NormalTok{]}
\ControlFlowTok{elif} \BuiltInTok{len}\NormalTok{(img\_resized.shape) }\OperatorTok{==} \DecValTok{2}\NormalTok{:}
    \ControlFlowTok{pass}
\ControlFlowTok{else}\NormalTok{:}
    \ControlFlowTok{raise} \PreprocessorTok{ValueError}\NormalTok{(}\SpecialStringTok{f"Unexpected image shape: }\SpecialCharTok{\{}\NormalTok{img\_resized}\SpecialCharTok{.}\NormalTok{shape}\SpecialCharTok{\}}\SpecialStringTok{"}\NormalTok{)}

\BuiltInTok{print}\NormalTok{(}\SpecialStringTok{f"Final image shape: }\SpecialCharTok{\{}\NormalTok{img\_resized}\SpecialCharTok{.}\NormalTok{shape}\SpecialCharTok{\}}\SpecialStringTok{ (should be 2D for grayscale)"}\NormalTok{)}

\CommentTok{\# Compute importance map using default parameters}
\NormalTok{importance\_map }\OperatorTok{=}\NormalTok{ compute\_importance(}
\NormalTok{    img\_resized,}
\NormalTok{    extreme\_downweight}\OperatorTok{=}\FloatTok{0.5}\NormalTok{,}
\NormalTok{    extreme\_threshold\_low}\OperatorTok{=}\FloatTok{0.2}\NormalTok{,}
\NormalTok{    extreme\_threshold\_high}\OperatorTok{=}\FloatTok{0.8}\NormalTok{,}
\NormalTok{    extreme\_sigma}\OperatorTok{=}\FloatTok{0.1}
\NormalTok{)}
\BuiltInTok{print}\NormalTok{(}\StringTok{"Importance map computed"}\NormalTok{)}
\end{Highlighting}
\end{Shaded}

\begin{verbatim}
Resized image from (6328, 4219) to (512, 341) for processing
Final image shape: (512, 341) (should be 2D for grayscale)
Importance map computed
\end{verbatim}

\subsection{Generating the Stippled
Image}\label{generating-the-stippled-image}

Now let's apply the stippling algorithm to create the blue noise
stippling pattern.

\subsubsection{Python}

\phantomsection\label{generate-stipple}
\begin{Shaded}
\begin{Highlighting}[]
\CommentTok{\# Generate stippling pattern}
\BuiltInTok{print}\NormalTok{(}\StringTok{"Generating blue noise stippling pattern..."}\NormalTok{)}
\NormalTok{stipple\_pattern, samples }\OperatorTok{=}\NormalTok{ void\_and\_cluster(}
\NormalTok{    img\_resized,}
\NormalTok{    percentage}\OperatorTok{=}\FloatTok{0.08}\NormalTok{,}
\NormalTok{    sigma}\OperatorTok{=}\FloatTok{0.9}\NormalTok{,}
\NormalTok{    content\_bias}\OperatorTok{=}\FloatTok{0.9}\NormalTok{,}
\NormalTok{    importance\_img}\OperatorTok{=}\NormalTok{importance\_map,}
\NormalTok{    noise\_scale\_factor}\OperatorTok{=}\FloatTok{0.1}
\NormalTok{)}

\BuiltInTok{print}\NormalTok{(}\SpecialStringTok{f"Generated }\SpecialCharTok{\{}\BuiltInTok{len}\NormalTok{(samples)}\SpecialCharTok{\}}\SpecialStringTok{ stipple points"}\NormalTok{)}
\BuiltInTok{print}\NormalTok{(}\SpecialStringTok{f"Stipple pattern shape: }\SpecialCharTok{\{}\NormalTok{stipple\_pattern}\SpecialCharTok{.}\NormalTok{shape}\SpecialCharTok{\}}\SpecialStringTok{"}\NormalTok{)}
\end{Highlighting}
\end{Shaded}

\begin{verbatim}
Generating blue noise stippling pattern...
Generated 13967 stipple points
Stipple pattern shape: (512, 341)
\end{verbatim}

\subsection{Displaying the Results}\label{displaying-the-results}

Let's visualize the original image, the importance map, and the stippled
version side by side for comparison.

\subsubsection{Python}

\begin{Shaded}
\begin{Highlighting}[]
\NormalTok{fig, axes }\OperatorTok{=}\NormalTok{ plt.subplots(}\DecValTok{1}\NormalTok{, }\DecValTok{3}\NormalTok{, figsize}\OperatorTok{=}\NormalTok{(}\DecValTok{7}\NormalTok{, }\DecValTok{4}\NormalTok{))}

\CommentTok{\# Display original image}
\NormalTok{axes[}\DecValTok{0}\NormalTok{].imshow(img\_resized, cmap}\OperatorTok{=}\StringTok{\textquotesingle{}gray\textquotesingle{}}\NormalTok{, vmin}\OperatorTok{=}\DecValTok{0}\NormalTok{, vmax}\OperatorTok{=}\DecValTok{1}\NormalTok{)}
\NormalTok{axes[}\DecValTok{0}\NormalTok{].axis(}\StringTok{\textquotesingle{}off\textquotesingle{}}\NormalTok{)}
\NormalTok{axes[}\DecValTok{0}\NormalTok{].set\_title(}\StringTok{\textquotesingle{}Original Image\textquotesingle{}}\NormalTok{, fontsize}\OperatorTok{=}\DecValTok{14}\NormalTok{, fontweight}\OperatorTok{=}\StringTok{\textquotesingle{}bold\textquotesingle{}}\NormalTok{, pad}\OperatorTok{=}\DecValTok{10}\NormalTok{)}

\CommentTok{\# Display importance map}
\NormalTok{axes[}\DecValTok{1}\NormalTok{].imshow(importance\_map, cmap}\OperatorTok{=}\StringTok{\textquotesingle{}gray\textquotesingle{}}\NormalTok{, vmin}\OperatorTok{=}\DecValTok{0}\NormalTok{, vmax}\OperatorTok{=}\DecValTok{1}\NormalTok{)}
\NormalTok{axes[}\DecValTok{1}\NormalTok{].axis(}\StringTok{\textquotesingle{}off\textquotesingle{}}\NormalTok{)}
\NormalTok{axes[}\DecValTok{1}\NormalTok{].set\_title(}\StringTok{\textquotesingle{}Importance Map\textquotesingle{}}\NormalTok{, fontsize}\OperatorTok{=}\DecValTok{14}\NormalTok{, fontweight}\OperatorTok{=}\StringTok{\textquotesingle{}bold\textquotesingle{}}\NormalTok{, pad}\OperatorTok{=}\DecValTok{10}\NormalTok{)}

\CommentTok{\# Display stippled image}
\NormalTok{axes[}\DecValTok{2}\NormalTok{].imshow(stipple\_pattern, cmap}\OperatorTok{=}\StringTok{\textquotesingle{}gray\textquotesingle{}}\NormalTok{, vmin}\OperatorTok{=}\DecValTok{0}\NormalTok{, vmax}\OperatorTok{=}\DecValTok{1}\NormalTok{)}
\NormalTok{axes[}\DecValTok{2}\NormalTok{].axis(}\StringTok{\textquotesingle{}off\textquotesingle{}}\NormalTok{)}
\NormalTok{axes[}\DecValTok{2}\NormalTok{].set\_title(}\StringTok{\textquotesingle{}Blue Noise Stippling\textquotesingle{}}\NormalTok{, fontsize}\OperatorTok{=}\DecValTok{14}\NormalTok{, fontweight}\OperatorTok{=}\StringTok{\textquotesingle{}bold\textquotesingle{}}\NormalTok{, pad}\OperatorTok{=}\DecValTok{10}\NormalTok{)}

\NormalTok{plt.tight\_layout()}
\NormalTok{plt.show()}
\end{Highlighting}
\end{Shaded}

\begin{figure}[H]

{\centering \pandocbounded{\includegraphics[keepaspectratio]{index_files/figure-pdf/display-results-output-1.pdf}}

}

\caption{Comparison of original image, importance map, and blue noise
stippling}

\end{figure}%

\subsection{Progressive Stippling
Animation}\label{progressive-stippling-animation}

This section creates a GIF showing how the stippled image looks as more
points are added sequentially. We'll use the already-computed stippling
points to generate frames at increments of 100 points.

\subsubsection{Python}

\phantomsection\label{progressive-stippling}
\begin{Shaded}
\begin{Highlighting}[]
\ImportTok{import}\NormalTok{ numpy }\ImportTok{as}\NormalTok{ np}
\ImportTok{from}\NormalTok{ PIL }\ImportTok{import}\NormalTok{ Image}
\ImportTok{import}\NormalTok{ matplotlib.pyplot }\ImportTok{as}\NormalTok{ plt}
\ImportTok{from}\NormalTok{ matplotlib.animation }\ImportTok{import}\NormalTok{ PillowWriter}

\CommentTok{\# Use the existing samples array from the already{-}computed stippling}
\BuiltInTok{print}\NormalTok{(}\SpecialStringTok{f"Using existing stippling with }\SpecialCharTok{\{}\BuiltInTok{len}\NormalTok{(samples)}\SpecialCharTok{\}}\SpecialStringTok{ points"}\NormalTok{)}
\BuiltInTok{print}\NormalTok{(}\SpecialStringTok{f"Image shape: }\SpecialCharTok{\{}\NormalTok{img\_resized}\SpecialCharTok{.}\NormalTok{shape}\SpecialCharTok{\}}\SpecialStringTok{"}\NormalTok{)}

\CommentTok{\# Create progressive frames by adding points sequentially}
\NormalTok{frame\_increment }\OperatorTok{=} \DecValTok{100}
\NormalTok{frames }\OperatorTok{=}\NormalTok{ []}
\NormalTok{point\_counts }\OperatorTok{=}\NormalTok{ []}

\CommentTok{\# Start with white background}
\NormalTok{h, w }\OperatorTok{=}\NormalTok{ img\_resized.shape}
\NormalTok{progressive\_stipple }\OperatorTok{=}\NormalTok{ np.ones\_like(img\_resized)}

\CommentTok{\# Add first point and save initial frame}
\ControlFlowTok{if} \BuiltInTok{len}\NormalTok{(samples) }\OperatorTok{\textgreater{}} \DecValTok{0}\NormalTok{:}
\NormalTok{    y0, x0, intensity0 }\OperatorTok{=} \BuiltInTok{int}\NormalTok{(samples[}\DecValTok{0}\NormalTok{, }\DecValTok{0}\NormalTok{]), }\BuiltInTok{int}\NormalTok{(samples[}\DecValTok{0}\NormalTok{, }\DecValTok{1}\NormalTok{]), samples[}\DecValTok{0}\NormalTok{, }\DecValTok{2}\NormalTok{]}
\NormalTok{    progressive\_stipple[y0, x0] }\OperatorTok{=} \FloatTok{0.0}
\NormalTok{    frames.append(progressive\_stipple.copy())}
\NormalTok{    point\_counts.append(}\DecValTok{1}\NormalTok{)}

\CommentTok{\# Add remaining points sequentially and save frames at increments}
\ControlFlowTok{for}\NormalTok{ i }\KeywordTok{in} \BuiltInTok{range}\NormalTok{(}\DecValTok{1}\NormalTok{, }\BuiltInTok{len}\NormalTok{(samples)):}
\NormalTok{    y, x }\OperatorTok{=} \BuiltInTok{int}\NormalTok{(samples[i, }\DecValTok{0}\NormalTok{]), }\BuiltInTok{int}\NormalTok{(samples[i, }\DecValTok{1}\NormalTok{])}
\NormalTok{    progressive\_stipple[y, x] }\OperatorTok{=} \FloatTok{0.0}  \CommentTok{\# Add black dot}
    
    \CommentTok{\# Save frame at increments (100, 200, 300, ...) and at the end}
    \ControlFlowTok{if}\NormalTok{ (i }\OperatorTok{+} \DecValTok{1}\NormalTok{) }\OperatorTok{\%}\NormalTok{ frame\_increment }\OperatorTok{==} \DecValTok{0} \KeywordTok{or}\NormalTok{ i }\OperatorTok{==} \BuiltInTok{len}\NormalTok{(samples) }\OperatorTok{{-}} \DecValTok{1}\NormalTok{:}
\NormalTok{        frames.append(progressive\_stipple.copy())}
\NormalTok{        point\_counts.append(i }\OperatorTok{+} \DecValTok{1}\NormalTok{)}

\BuiltInTok{print}\NormalTok{(}\SpecialStringTok{f"Generated }\SpecialCharTok{\{}\BuiltInTok{len}\NormalTok{(frames)}\SpecialCharTok{\}}\SpecialStringTok{ frames"}\NormalTok{)}
\BuiltInTok{print}\NormalTok{(}\SpecialStringTok{f"Point counts: }\SpecialCharTok{\{}\NormalTok{point\_counts}\SpecialCharTok{\}}\SpecialStringTok{"}\NormalTok{)}
\end{Highlighting}
\end{Shaded}

\begin{verbatim}
Using existing stippling with 13967 points
Image shape: (512, 341)
Generated 141 frames
Point counts: [1, 100, 200, 300, 400, 500, 600, 700, 800, 900, 1000, 1100, 1200, 1300, 1400, 1500, 1600, 1700, 1800, 1900, 2000, 2100, 2200, 2300, 2400, 2500, 2600, 2700, 2800, 2900, 3000, 3100, 3200, 3300, 3400, 3500, 3600, 3700, 3800, 3900, 4000, 4100, 4200, 4300, 4400, 4500, 4600, 4700, 4800, 4900, 5000, 5100, 5200, 5300, 5400, 5500, 5600, 5700, 5800, 5900, 6000, 6100, 6200, 6300, 6400, 6500, 6600, 6700, 6800, 6900, 7000, 7100, 7200, 7300, 7400, 7500, 7600, 7700, 7800, 7900, 8000, 8100, 8200, 8300, 8400, 8500, 8600, 8700, 8800, 8900, 9000, 9100, 9200, 9300, 9400, 9500, 9600, 9700, 9800, 9900, 10000, 10100, 10200, 10300, 10400, 10500, 10600, 10700, 10800, 10900, 11000, 11100, 11200, 11300, 11400, 11500, 11600, 11700, 11800, 11900, 12000, 12100, 12200, 12300, 12400, 12500, 12600, 12700, 12800, 12900, 13000, 13100, 13200, 13300, 13400, 13500, 13600, 13700, 13800, 13900, 13967]
\end{verbatim}

Now let's create the GIF animation:

\subsubsection{Python}

\begin{figure}[H]

{\centering \pandocbounded{\includegraphics[keepaspectratio]{progressive_stippling.gif}}

}

\caption{Progressive stippling animation showing the sequential build-up
of points. Each frame represents an increment of 100 points,
demonstrating how the blue noise stippling pattern develops as more
points are added.}

\end{figure}%

\subsection{Challenge Requirements 📋}\label{challenge-requirements-1}

\subsubsection{Your Task}\label{your-task}

\textbf{Part 1 (This Challenge):} Reproduce the blue noise stippling
process to create:

\begin{enumerate}
\def\labelenumi{\arabic{enumi}.}
\tightlist
\item
  \textbf{A stippled version of your chosen image} - Use the algorithm
  demonstrated above to create a stippled version of an image of your
  choice
\item
  \textbf{A progressive stippling GIF animation} - Create an animated
  GIF showing how the stippling pattern builds up as points are added
\item
  \textbf{GitHub Pages site} - Post both the stippled image and GIF to a
  GitHub Pages site with:

  \begin{itemize}
  \tightlist
  \item
    Appropriate captions explaining what the images show
  \item
    A brief explanation of blue noise stippling and why it creates
    aesthetically pleasing patterns
  \end{itemize}
\end{enumerate}

\textbf{Part 2 (November 18th):} We'll use your stippled images to
create a statistical meme about selection bias and missing data. More
details will be provided in Part 2.

\subsubsection{Minimum Requirements for Any
Points}\label{minimum-requirements-for-any-points}

\begin{enumerate}
\def\labelenumi{\arabic{enumi}.}
\item
  \textbf{Fork and Clone Repository:} Use the starter repository (if
  provided) or create your own GitHub repository for this challenge
\item
  \textbf{Reproduce the Stippling Process:}

  \begin{itemize}
  \tightlist
  \item
    Implement the blue noise stippling algorithm (Python recommended, R
    acceptable)
  \item
    Create a stippled version of your chosen image
  \item
    Generate a progressive stippling GIF animation
  \end{itemize}
\item
  \textbf{GitHub Pages Setup:}

  \begin{itemize}
  \tightlist
  \item
    Create a GitHub Pages site showcasing your work
  \item
    Include the stippled image with appropriate caption
  \item
    Include the progressive GIF with appropriate caption
  \item
    Add a brief explanation of blue noise stippling
  \end{itemize}
\item
  \textbf{Code Organization:}

  \begin{itemize}
  \tightlist
  \item
    Well-organized, readable code
  \item
    Comments explaining key steps
  \item
    Use just one language (Python or R) - delete the other language
    sections
  \end{itemize}
\end{enumerate}

\subsubsection{Grading Rubric 🎓}\label{grading-rubric}

\textbf{75\% Grade Requirements:} - Successfully reproduce the stippling
algorithm - Create a stippled image and progressive GIF - Basic GitHub
Pages site with images and captions

\textbf{85\% Grade Requirements:} - All of the above, plus: - Clear
explanation of how blue noise stippling works - Discussion of why the
algorithm creates aesthetically pleasing patterns - Professional
presentation of results

\textbf{95\% Grade Requirements:} - All of the above, plus: -
Experimentation with different parameters (percentage, sigma, etc.) -
Discussion of how parameter choices affect the final result - Comparison
of different parameter settings

\textbf{100\% Grade Requirements:} - All of the above, plus: -
Exceptional presentation and documentation - Creative use of the
stippling technique - Insightful analysis of the algorithm's behavior -
Professional-quality GitHub Pages site

\subsection{Getting Started Tips 🚀}\label{getting-started-tips}

\begin{tcolorbox}[enhanced jigsaw, leftrule=.75mm, colback=white, opacityback=0, arc=.35mm, left=2mm, rightrule=.15mm, coltitle=black, titlerule=0mm, breakable, bottomrule=.15mm, colbacktitle=quarto-callout-note-color!10!white, bottomtitle=1mm, toptitle=1mm, colframe=quarto-callout-note-color-frame, title=\textcolor{quarto-callout-note-color}{\faInfo}\hspace{0.5em}{🎯 Navy SEALs Motto}, toprule=.15mm, opacitybacktitle=0.6]

\begin{quote}
``Slow is Smooth and Smooth is Fast''
\end{quote}

\emph{Take your time to understand the stippling algorithm, plan your
approach carefully, and execute with precision. Rushing through this
challenge will only lead to errors and confusion.}

\end{tcolorbox}

\begin{tcolorbox}[enhanced jigsaw, leftrule=.75mm, colback=white, opacityback=0, arc=.35mm, left=2mm, rightrule=.15mm, coltitle=black, titlerule=0mm, breakable, bottomrule=.15mm, colbacktitle=quarto-callout-warning-color!10!white, bottomtitle=1mm, toptitle=1mm, colframe=quarto-callout-warning-color-frame, title=\textcolor{quarto-callout-warning-color}{\faExclamationTriangle}\hspace{0.5em}{💾 Important: Save Your Work Frequently!}, toprule=.15mm, opacitybacktitle=0.6]

\textbf{Before you start:} Make sure to commit your work often using the
Source Control panel in Cursor (Ctrl+Shift+G or Cmd+Shift+G). This
prevents the AI from overwriting your progress and ensures you don't
lose your work.

\textbf{Commit after each major step:} - After implementing the
importance map function - After implementing the stippling algorithm -
After creating your stippled image - After creating your GIF animation -
After setting up GitHub Pages

\emph{Remember: Frequent commits are your safety net!}

\end{tcolorbox}

\subsection{Submission Checklist ✅}\label{submission-checklist}

\textbf{Minimum Requirements (Required for Any Points):}

\begin{itemize}
\tightlist
\item[$\square$]
  Forked/cloned repository for this challenge
\item[$\square$]
  Implemented blue noise stippling algorithm
\item[$\square$]
  Created stippled version of your image
\item[$\square$]
  Generated progressive stippling GIF animation
\item[$\square$]
  Created GitHub Pages site
\item[$\square$]
  Posted stippled image with caption
\item[$\square$]
  Posted progressive GIF with caption
\item[$\square$]
  Added brief explanation of blue noise stippling
\end{itemize}

\textbf{75\% Grade Requirements:}

\begin{itemize}
\tightlist
\item[$\square$]
  All minimum requirements met
\item[$\square$]
  Code is well-organized and readable
\item[$\square$]
  GitHub Pages site is functional
\end{itemize}

\textbf{85\% Grade Requirements:}

\begin{itemize}
\tightlist
\item[$\square$]
  All 75\% requirements met
\item[$\square$]
  Clear explanation of how blue noise stippling works
\item[$\square$]
  Discussion of why the algorithm creates aesthetically pleasing
  patterns
\end{itemize}

\textbf{95\% Grade Requirements:}

\begin{itemize}
\tightlist
\item[$\square$]
  All 85\% requirements met
\item[$\square$]
  Experimentation with different parameters
\item[$\square$]
  Discussion of parameter effects
\end{itemize}

\textbf{100\% Grade Requirements:}

\begin{itemize}
\tightlist
\item[$\square$]
  All 95\% requirements met
\item[$\square$]
  Exceptional presentation and documentation
\item[$\square$]
  Creative use of the stippling technique
\item[$\square$]
  Professional-quality GitHub Pages site
\end{itemize}




\end{document}
